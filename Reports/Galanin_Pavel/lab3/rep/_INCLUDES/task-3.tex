\subparagraph{Задание 3} \textbf{Условие}:

Вывести сообщение с помощью команды \verb|echo| в канал ошибок.
Создать файл \verb|myscript|:

\begin{verbatim}
    #!/bin/sh
    echo stdout
    echo stderr>&2
    exit 0
\end{verbatim}

Запустить его:
\begin{itemize}
    \item без перенаправления (sh myscript)
    \item перенаправив стандартный вывод в файл, просмотреть содержимое файла (sh myscript \verb|>| file1)
    \item перенаправить стандартный канал ошибок в существующий и несуществующий файлы с помощью символов \verb|>| и \verb|>>|
    \item перенаправив стандартный вывод в файл 1, стандартный канал ошибок - в файл 2
    \item перенаправив стандартный вывод и стандартный канал ошибок в файл 3
    \item перенаправив стандартный вывод в файл 4 с помощью символа \verb|>|, а стандартный канал ошибок в файл 4 с помощью символа \verb|>>|
\end{itemize}

Объяснить результаты.

\textbf{Решение}:

Создаем файл.

\begin{BashBox}
nano myscipt
\end{BashBox}

Пишем текст.

\begin{BashBox}
#!/bin/sh
echo stdout
echo stderr>&2
exit 0
\end{BashBox}

Сохраняем: \textbf{Ctrl} + \textbf{X}, \textbf{Y}, \textbf{Enter}.

Запускаем скрипт.

\begin{BashBox}
sh myscript
\end{BashBox}

\begin{OutBox}
stdout
stderr
\end{OutBox}

Перенаправляем стандартный вывод в файл.

\begin{BashBox}
sh myscript > file1
\end{BashBox}

\begin{OutBox}
stderr
\end{OutBox}

Просматриваем содержимое файла.

\begin{BashBox}
cat file1
\end{BashBox}

\begin{OutBox}
stdout
\end{OutBox}

Перенаправляем стандартный поток в несуществующий файл с помощью символа больше \verb|>|.

\begin{BashBox}
sh myscript > 3_file-1.txt
\end{BashBox}

\begin{OutBox}
stderr
\end{OutBox}

\begin{BashBox}
cat 3_file-1.txt
\end{BashBox}

\begin{OutBox}
stdout
\end{OutBox}

Перенаправляем стандартный поток в несуществующий файл с помощью двух символов больше \verb|>>|.

\begin{BashBox}
sh myscript >> 3_file-2.txt
\end{BashBox}

\begin{OutBox}
stderr
\end{OutBox}

\begin{BashBox}
cat 3_file-2.txt
\end{BashBox}

\begin{OutBox}
stdout
\end{OutBox}

Перенаправляем стандартный поток в существующий файл с помощью символа больше \verb|>|.

\begin{BashBox}
sh myscript > 3_file-1.txt
\end{BashBox}

\begin{OutBox}
stderr
\end{OutBox}

\begin{BashBox}
cat 3_file-1.txt
\end{BashBox}

\begin{OutBox}
stdout
\end{OutBox}

Перенаправляем стандартный поток в существующий файл с помощью двух символов больше \verb|>>|.

\begin{BashBox}
sh myscript >> 3_file-2.txt
\end{BashBox}

\begin{OutBox}
stderr
\end{OutBox}

\begin{BashBox}
cat 3_file-2.txt
\end{BashBox}

\begin{OutBox}
stdout
stdout
\end{OutBox}

Перенаправляем стандартный вывод в файл 1.

\begin{BashBox}
sh myscript 1>1.txt
\end{BashBox}

\begin{OutBox}
stderr
\end{OutBox}

\begin{BashBox}
cat 1.txt
\end{BashBox}

\begin{OutBox}
stdout
\end{OutBox}

Перенаправляем стандартный канал ошибок в файл 2.

\begin{BashBox}
sh myscript 2>2.txt
\end{BashBox}

\begin{OutBox}
stdout
\end{OutBox}

\begin{BashBox}
cat 2.txt
\end{BashBox}

\begin{OutBox}
stderr
\end{OutBox}

Перенаправляем стандартный вывод и стандартный канал ошибок в файл 3.

\begin{BashBox}
sh myscript 1>3.txt 2>>3.txt
cat 3.txt
\end{BashBox}

\begin{OutBox}
stdout
stderr
\end{OutBox}

Перенаправляем стандартный вывод в файл 4 с помощью символа \verb|>|, а стандартный канал ошибок в файл 4 с помощью символа \verb|>>|.

\begin{BashBox}
sh myscript 1>4.txt 2>>4.txt
cat 4.txt
\end{BashBox}

\begin{OutBox}
stdout
stderr
\end{OutBox}
