\subparagraph{Задание 4} \textbf{Условие}:

Изучите команду \verb|chmod|.
Создайте в \verb|домашнем каталоге| любые четыре файла, установите при помощи восмеричных масок на каждый из них в отдельности следующие права:
\begin{itemize}
\item для себя все права, для группы и остальных - никаких;
\item для себя чтение и запись, для группы чтение, для остальных - все;
\item для себя исполнение и запись, для группы никаких, для остальных чтение;
\item для себя запись, для группы все, для остальных - только запись.
\end{itemize}

\textbf{Решение}:

\begin{table}[h!]
    \centering
    \caption{Восьмиричная маска для файла 1}
    \begin{tabular}{ | l | l | l | l | l | }
        \hline
        Описание            & Тип   & Владелец  & Группы    & Остальные \\ \hline
        \hline
        Что нужно           & файл  & все права & нет прав  & нет прав  \\ \hline
        Символьная запись 	& -     & rwx 	    & --- 	    & ---       \\ \hline
        Двоичная запись     & 	    & 111 	    & 000 	    & 000       \\ \hline
        Восьмиричное запись & 	    & 7 	    & 0 	    & 0         \\ \hline
    \end{tabular}
\end{table}

\begin{BashBox}
    touch file1.txt
    chmod 700 file1.txt
    ls -l file1.txt
\end{BashBox}

\begin{OutBox}
    -rwx------ 1 pavel-innokentevich-galanin pavel-innokentevich-galanin 0 Mar  1 23:59 file1.txt
\end{OutBox}

\begin{table}[h!]
    \centering
    \caption{Восьмиричная маска для файла 2}
    \begin{tabular}{ | l | l | l | l | l | }
        \hline
        Описание            & Тип   & Владелец          & Группы    & Остальные \\ \hline
        \hline
        Что нужно           & файл  & чтение и запись   & чтение    & все права \\ \hline
        Символьная запись 	& -     & rw- 	            & r-- 	    & rwx       \\ \hline
        Двоичная запись     & 	    & 110 	            & 100 	    & 111       \\ \hline
        Восьмиричное запись & 	    & 6 	            & 4 	    & 7         \\ \hline
    \end{tabular}
\end{table}

\begin{BashBox}
    touch file2.txt
    chmod 647 file2.txt
    ls -l file2.txt
\end{BashBox}

\begin{OutBox}
    -rw-r--rwx 1 pavel-innokentevich-galanin pavel-innokentevich-galanin 0 Mar  1 23:59 file2.txt
\end{OutBox}

\begin{table}[h!]
    \centering
    \caption{Восьмиричная маска для файла 3}
    \begin{tabular}{ | l | l | l | l | l | }
        \hline
        Описание            & Тип   & Владелец              & Группы    & Остальные \\ \hline
        \hline
        Что нужно           & файл  & исполнение и запись   & нет прав  & чтение    \\ \hline
        Символьная запись 	& -     & -wx 	                & --- 	    & r--       \\ \hline
        Двоичная запись     & 	    & 011 	                & 000 	    & 100       \\ \hline
        Восьмиричное запись & 	    & 3 	                & 0 	    & 4         \\ \hline
    \end{tabular}
\end{table}

\begin{BashBox}
    touch file3.txt
    chmod 304 file3.txt
    ls -l file3.txt
\end{BashBox}

\begin{OutBox}
    --wx---r-- 1 pavel-innokentevich-galanin pavel-innokentevich-galanin 0 Mar  1 23:59 file3.txt
\end{OutBox}

\begin{table}[h!]
    \centering
    \caption{Восьмиричная маска для файла 4}
    \begin{tabular}{ | l | l | l | l | l | }
        \hline
        Описание            & Тип   & Владелец  & Группы    & Остальные \\ \hline
        \hline
        Что нужно           & файл  & запись    & все прав  & запись    \\ \hline
        Символьная запись 	& -     & -w- 	    & rwx 	    & -w-       \\ \hline
        Двоичная запись     & 	    & 010 	    & 111 	    & 010       \\ \hline
        Восьмиричное запись & 	    & 2	        & 7 	    & 2         \\ \hline
    \end{tabular}
\end{table}

\begin{BashBox}
    touch file4.txt
    chmod 272 file4.txt
    ls -l file4.txt
\end{BashBox}

\begin{OutBox}
    --w-rwx-w- 1 pavel-innokentevich-galanin pavel-innokentevich-galanin 0 Mar  2 00:00 file4.txt
\end{OutBox}
