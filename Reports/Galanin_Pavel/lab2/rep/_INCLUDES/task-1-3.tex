\subparagraph{Задание 3} \textbf{Условие}:

Создать жесткую и символьную ссылки на файл. С помощью команды ls
просмотреть inod файла и ссылок. Объяснить результат.

\textbf{Решение}:

Изучаем назначение и ключи команды ln

\begin{BashBox}
    man ln
\end{BashBox}

Создаем файл

\begin{BashBox}
    nano 1-2_file.txt
\end{BashBox}

Пишем текст

\begin{BashBox}
    The storm broke today
    and the sun came out.
\end{BashBox}

Сохраняем нажатием: \textbf{Ctrl} + \textbf{X}, \textbf{Y}, \textbf{Enter}.

Создаём символьную ссылку на файл

\begin{BashBox}
    ln -s 1-2_file.txt 1-2_file-link.txt
\end{BashBox}

Просматриваем содержимое файла с помощью сслыки

\begin{BashBox}
    cat 1-2_file-link.txt
\end{BashBox}

\begin{OutBox}
    The storm broke today
    and the sun came out.
\end{OutBox}

Удаляем файл

\begin{BashBox}
    rm 1-2_file.txt
\end{BashBox}

Просматриваем содержимое файла с помощью сслыки

\begin{BashBox}
    cat 1-2_file-link.txt
\end{BashBox}

\begin{OutBox}
    cat: 1-2_file-link.txt: No such file or directory
\end{OutBox}

Создаем каталог

\begin{BashBox}
    mkdir 1-2_katalog
\end{BashBox}

Создаем жёсткую ссылку на каталог

\begin{BashBox}
    ln -s 1-2_katalog 1-2_katalog-link
\end{BashBox}
