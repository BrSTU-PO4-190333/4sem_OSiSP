\subsection{Просмотр каталога (ls) системного и начального}

\subparagraph{1) Что нужно было сделать?}

Перейти в системный каталог. Просмотреть его содержимое. Просмотреть содержимое начального каталога. Вернуться в начальный каталог.

\subparagraph{2) Как это сделали?}

\begin{MyVerbatimCode}[label=Debian terminal]
pavel-innokentevich-galanin@aspire-one-725:~$ ls
Apps     Documents  Music     Public     Videos    katalog2
Desktop  Downloads  Pictures  Templates  katalog1  www
pavel-innokentevich-galanin@aspire-one-725:~$ cd /
pavel-innokentevich-galanin@aspire-one-725:/$ ls
bin    dev   lib         media  proc  sbin      sys        usr
boot   etc   lib64       mnt    root  srv       timeshift  var
cdrom  home  lost+found  opt    run   swapfile  tmp
pavel-innokentevich-galanin@aspire-one-725:/$ cd
pavel-innokentevich-galanin@aspire-one-725:~$ ls
Apps     Documents  Music     Public     Videos    katalog2
Desktop  Downloads  Pictures  Templates  katalog1  www    
\end{MyVerbatimCode}

\subparagraph{3) Что получилось?}

Командой cd сделали переход по директории.
Команда cd и палочка - переход в системную директорию. Команда cd и тильда или пустота - переход в директорию /home/username/. Командой ls просмотрели содержимое директории.