\subsection*{Модифицированная версия (две программы))}

В программе номер один вводим стороны треугольника. В программе номер один Открываем очередь. В программе номер один строны треугольника отправляем в очередь.

В программе номер два открываем очередь с тем же именем и принимаем сообщения (очередь закрываем). В программе номер два происходит вычисление площади треугольника по формуле Герона. В программе номер два открываем очередь и отправляем сообщение с площадью.

В программе номер один открывем очередь и принимаем сообщение с площадью (очередь закрываем).

\newpage

\lstinputlisting[
    name=Makefile,
    language=Make,
    basicstyle=\ttfamily\scriptsize,
]
{../../src/option5-mod/Makefile}

\lstinputlisting[
    name=app1.c,
    language=C++,
    basicstyle=\ttfamily\scriptsize,
]
{../../src/option5-mod/src/app1.c}

\lstinputlisting[
    name=app2.c,
    language=C++,
    basicstyle=\ttfamily\scriptsize,
]
{../../src/option5-mod/src/app2.c}

\lstinputlisting[
    name=message_queue.h,
    language=C++,
    basicstyle=\ttfamily\scriptsize,
]
{../../src/option5-mod/src/message_queue/message_queue.h}

\lstinputlisting[
    name=message_queue.c,
    language=C++,
    basicstyle=\ttfamily\scriptsize,
]
{../../src/option5-mod/src/message_queue/message_queue.c}

\lstinputlisting[
    name=my_ftoa.h,
    language=C++,
    basicstyle=\ttfamily\scriptsize,
]
{../../src/option5-mod/src/my_ftoa/my_ftoa.h}

\lstinputlisting[
    name=my_ftoa.c,
    language=C++,
    basicstyle=\ttfamily\scriptsize,
]
{../../src/option5-mod/src/my_ftoa/my_ftoa.c}

\lstinputlisting[
    name=print_real_time.h,
    language=C++,
    basicstyle=\ttfamily\scriptsize,
]
{../../src/option5-mod/src/print_real_time/print_real_time.h}

\lstinputlisting[
    name=print_real_time.c,
    language=C++,
    basicstyle=\ttfamily\scriptsize,
]
{../../src/option5-mod/src/print_real_time/print_real_time.c}

%

\begin{lstlisting}[language=Terminal, basicstyle=\ttfamily\scriptsize]
make compile_app1
\end{lstlisting}

\begin{lstlisting}[language=Out, basicstyle=\ttfamily\scriptsize]
mkdir -p bin
cd bin; gcc ./../src/app1.c ./../src/message_queue/message_queue.c ./../src/my_ftoa/my_ftoa.c ./../src/print_real_time/print_real_time.c -lrt -o app1.out
\end{lstlisting}

\begin{lstlisting}[language=Terminal, basicstyle=\ttfamily\scriptsize]
make run_app1
\end{lstlisting}

\begin{lstlisting}[language=Out, basicstyle=\ttfamily\scriptsize]
cd bin; ./app1.out
a = 4
b = 5
c = 6
01:16:53
        "queue": "/abc" - открывается очередь
        mqd_t 3 - очередь открыта успешно
01:16:53
        "message": "4.000000" - сообщение отправляется
        mqd_t 3 - сообщение отправлено успешно
01:16:53
        "message": "5.000000" - сообщение отправляется
        mqd_t 3 - сообщение отправлено успешно
01:16:53
        "message": "6.000000" - сообщение отправляется
        mqd_t 3 - сообщение отправлено успешно
01:16:53
        "queue": "/s" - открывается очередь
        mqd_t 4 - очередь открыта успешно
01:16:53
        mqd_t 4 - получаем сообщение
        "message": "9.921567" - сообщение получено успешно
01:17:07
        "queue": "/s" - очередь закрывается
        "queue": "/s" - очередь закрыта успешно
\end{lstlisting}

\newpage

\begin{lstlisting}[language=Terminal, basicstyle=\ttfamily\scriptsize]
make compile_app2
\end{lstlisting}
    
\begin{lstlisting}[language=Out, basicstyle=\ttfamily\scriptsize]
mkdir -p bin
cd bin; gcc ./../src/app2.c ./../src/message_queue/message_queue.c ./../src/my_ftoa/my_ftoa.c ./../src/print_real_time/print_real_time.c -lrt -lm -o app2.out
\end{lstlisting}
    
\begin{lstlisting}[language=Terminal, basicstyle=\ttfamily\scriptsize]
make run_app2
\end{lstlisting}
    
\begin{lstlisting}[language=Out, basicstyle=\ttfamily\scriptsize]
cd bin; ./app2.out
01:17:07
        "queue": "/abc" - открывается очередь
        mqd_t 3 - очередь открыта успешно
01:17:07
        mqd_t 3 - получаем сообщение
        "message": "4.000000" - сообщение получено успешно
01:17:07
        mqd_t 3 - получаем сообщение
        "message": "5.000000" - сообщение получено успешно
01:17:07
        mqd_t 3 - получаем сообщение
        "message": "6.000000" - сообщение получено успешно
01:17:07
        "queue": "/abc" - очередь закрывается
        "queue": "/abc" - очередь закрыта успешно

S = sqrt(
        7.500000 *
        * (7.500000 - 4.000000) *
        * (7.500000 - 5.000000) *
        * (7.500000 - 6.000000)
) = 9.921567

01:17:07
        "queue": "/s" - открывается очередь
        mqd_t 4 - очередь открыта успешно
01:17:07
        "message": "9.921567" - сообщение отправляется
        mqd_t 4 - сообщение отправлено успешно
\end{lstlisting}