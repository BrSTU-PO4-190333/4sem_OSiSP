\paragraph{Условие:} \hspace{0pt}

\begin{table}[h]
    \centering{}
    \caption{Варианты}
    \label{table:name}

    \begin{tabular}{|l|l|p{10cm}|}
        \hline
        Вариант &   Средство взаимодействия &   Действия    \\  \hline
        \hline
        ...     &   ...                     &   ...         \\  \hline
        5       &   Очереди сообщений       & Родитель передает потомку три стороны треугольника, потомок возвращает его площадь    \\  \hline
        ...     &   ...                     &   ...         \\  \hline
    \end{tabular}
\end{table}

\paragraph{Решение:} \hspace{0pt}

\noindent\begin{minipage}{.47\textwidth}
    \lstinputlisting[name=CMakeLists.txt,]
    {../../src/option5/CMakeLists.txt}
\end{minipage}
\hfill
\begin{minipage}{.47\textwidth}
    \begin{lstlisting}[language=Terminal,]
tree --charset ascii
\end{lstlisting}
    \begin{lstlisting}[language=Out,]
.
|-- build
|-- CMakeLists.txt
|-- README.md
|-- README-ru.md
`-- src
    |-- main.c
    |-- message_queue
    |   |-- message_queue.c
    |   `-- message_queue.h
    `-- my_ftoa
        |-- my_ftoa.c
        `-- my_ftoa.h
\end{lstlisting}
\end{minipage}

\newpage

\lstinputlisting[language=C++, name=main.c,]
{../../src/option5/src/main.c}

\newpage

\lstinputlisting[language=C++, name=message_queue.c,]
{../../src/option5/src/message_queue/message_queue.c}

\lstinputlisting[language=C++, name=my_ftoa.c,]
{../../src/option5/src/my_ftoa/my_ftoa.c}

\newpage

\begin{lstlisting}[language=Terminal,]
mkdir -p build
cd build
cmake ..
cmake --build .
./main
\end{lstlisting}

\begin{lstlisting}[language=Out,]
::::: start [process 0] PID = 14003 PPID = 12430 :::::
a = 2
b = 2
c = 5
Not triangle
2.000000 * 2.000000 not > 5.000000
4.000000 not > 5.000000

a = 3
b = 4
c = 5
Is triagle
3.000000 * 4.000000 > 5.000000
7.000000 > 5.000000

        = = = = = start part 1 = = = = =
        ::::: start [parent process] PID = 14003 PPID = 12430 :::::
"queue": "/abc" - открывается очередь
mqd_t 3 - очередь открыта успешно
"message": "3.000000" - сообщение отправляется
mqd_t 3 - сообщение отправлено успешно
"message": "4.000000" - сообщение отправляется
mqd_t 3 - сообщение отправлено успешно
"message": "5.000000" - сообщение отправляется
mqd_t 3 - сообщение отправлено успешно
        ::::: kill [parent process] PID = 14003 PPID = 12430 :::::
        = = = = = end part 1 = = = = =
        = = = = = my_handler = = = = =
        = = = = = start part 2 = = = = =
        ::::: start [child process] PID = 14007 PPID = 14003 :::::
"queue": "/abc" - открывается очередь
mqd_t 3 - очередь открыта успешно
mqd_t 3 - получаем сообщение
"message": "3.000000" - сообщение получено успешно
mqd_t 3 - получаем сообщение
"message": "4.000000" - сообщение получено успешно
mqd_t 3 - получаем сообщение
"message": "5.000000" - сообщение получено успешно
"queue": "/abc" - очередь закрывается
"queue": "/abc" - очередь закрыта успешно

S = sqrt(
        6.000000 *
        * (6.000000 - 3.000000) *
        * (6.000000 - 4.000000) *
        * (6.000000 - 5.000000)
) = 6.000000

"queue": "/s" - открывается очередь
mqd_t 4 - очередь открыта успешно
"message": "6.000000" - сообщение отправляется
mqd_t 4 - сообщение отправлено успешно
        ::::: kill [child process] PID = 14007 PPID = 14003 :::::
        = = = = = end part 2 = = = = =
        = = = = = my_handler = = = = =
        = = = = = start part 3 = = = = =
        ::::: respawn [parent process] PID = 14003 PPID = 12430 :::::
"queue": "/s" - открывается очередь
mqd_t 4 - очередь открыта успешно
mqd_t 4 - получаем сообщение
"message": "6.0000000�H�" - сообщение получено успешно
"queue": "/s" - очередь закрывается
"queue": "/s" - очередь закрыта успешно
        ::::: end [parent process] PID = 14003 PPID = 12430 :::::
        = = = = = end part 3 = = = = =
        = = = = = start part 4 = = = = =
        ::::: respawn [child process] PID = 14007 PPID = 14003 :::::
        ::::: end [child process] PID = 14007 PPID = 1 :::::
        = = = = = end part 4 = = = = =
::::: end [process 0] PID = 14007 PPID = 1 :::::
\end{lstlisting}