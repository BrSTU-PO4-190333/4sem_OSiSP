\subparagraph{Задание 1} \textbf{Условие}:

Вывести любое сообщение с помощью команды echo перенаправив вывод:

\begin{itemize}
    \item в несуществующий файл с помощью символа \verb|>|
    \item в несуществующий файл с помощью символа \verb|>>|
    \item в существующий файл с помощью символа \verb|>|
    \item в существующий файл с помощью символа \verb|>>|
\end{itemize}

Объяснить результаты.

\textbf{Решение}:

\begin{BashBox}
echo "Message1" > 1_file-1.txt
cat 1_file-1.txt
\end{BashBox}

\begin{OutBox}
Message1
\end{OutBox}

\begin{BashBox}
echo "Message2" >> 1_file-2.txt
cat 1_file-2.txt
\end{BashBox}

\begin{OutBox}
Message2
\end{OutBox}

\begin{BashBox}
echo "Message3" > 1_file-1.txt
cat 1_file-1.txt
\end{BashBox}

\begin{OutBox}
Message3
\end{OutBox}

\begin{BashBox}
echo "Message4" >> 1_file-2.txt
cat 1_file-2.txt
\end{BashBox}

\begin{OutBox}
Message2
Message4
\end{OutBox}

\textbf{Вывод}:

\begin{itemize}
    \item Если нет файла, то символ больше (\verb|>|) создаст файл с сообщением.
    \item Если нет файла, то два символа больше (\verb|>>|) создадут файл с сообщением.
    \item Если файл существует, то символ больше (\verb|>|) перезапишет файл.
    \item Если файл существует, то два символа больше (\verb|>>|) добавить новую строчку в файл.
\end{itemize}
