\subparagraph{Задание 1} \textbf{Условие}:

Изучить назначение и ключи команды ln.

\begin{itemize}
    \item создать жесткую ссылку на файл. Просмотреть содержимое файла, используя
ссылку. Удалить файл. Просмотреть содержимое файла. Объяснить результат;
создать жесткую ссылку на каталог. Объяснить результат;
    \item создать жесткую ссылку на каталог. Объяснить результат;
\end{itemize}

\textbf{Решение}:

Изучаем назначение и ключи команды ln

\begin{BashBox}
    man ln
\end{BashBox}

Создаем файл

\begin{BashBox}
    nano 1-1_file.txt
\end{BashBox}

Пишем текст

\begin{BashBox}
    The storm broke today
    and the sun came out.
\end{BashBox}

Сохраняем нажатием: \textbf{Ctrl} + \textbf{X}, \textbf{Y}, \textbf{Enter}.

Создаём жёсткую ссылку на файл

\begin{BashBox}
    ln 1-1_file.txt 1-1_file-link.txt
\end{BashBox}

Просматриваем содержимое файла с помощью сслыки

\begin{BashBox}
    cat 1-1_file-link.txt
\end{BashBox}

\begin{OutBox}
    The storm broke today
    and the sun came out.
\end{OutBox}

Удаляем файл

\begin{BashBox}
    rm 1-1_file.txt
\end{BashBox}

Просматриваем содержимое файла с помощью сслыки

\begin{BashBox}
    cat 1-1_file-link.txt
\end{BashBox}

\begin{OutBox}
    The storm broke today
    and the sun came out.
\end{OutBox}

Создаем каталог

\begin{BashBox}
    mkdir 1-1_katalog
\end{BashBox}

Создаем жёсткую ссылку на каталог

\begin{BashBox}
    ln 1-1_katalog 1-1_katalog-link
\end{BashBox}

\begin{OutBox}
    ln: 1-1_katalog: hard link not allowed for directory
\end{OutBox}
