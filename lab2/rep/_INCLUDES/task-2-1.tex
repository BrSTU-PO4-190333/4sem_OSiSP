\subparagraph{Задание 1} \textbf{Условие}:

Изучите при помощи man опцию \verb|-l| команды \verb|ls|.
Просмотрите права каталогов \verb|/etc|, \verb|/bin| и \verb|домашнего каталога|. Просмотрите права файлов, содержащиеся в этих каталогов. Выявите тенденции (файлов с какими правами в каких каталогах больше).
Сделайте вывод.

\textbf{Решение}:

Изучаем при помощи man опцию -l команды ls.

\begin{BashBox}
    man ls
\end{BashBox}

\begin{OutBox}
        -l     use a long listing format
\end{OutBox}

Просматриваем права каталога /etc

\begin{BashBox}
    ls -ld /etc
\end{BashBox}

\begin{OutBox}
    drwxr-xr-x 140 root root 12288 Mar  1 16:21 /etc
\end{OutBox}

\begin{table}[h!]
    \centering
    \begin{tabular}{ | l | l | l | l | l | }
        \hline
        Описание            & Тип       & Владелец  & Группы            & Остальные       \\ \hline
        \hline
        Символьная  запись  & d         & rwx       & r-x               & r-x             \\ \hline
        Что значит          & директория& все права & доступ на чтение  & доступ на чтение\\ \hline
    \end{tabular}
\end{table}

Просматриваем права каталога /bin

\begin{BashBox}
    ls -ld /bin
\end{BashBox}

\begin{OutBox}
    lrwxrwxrwx 1 root root 7 Feb 25 21:42 /bin -> usr/bin
\end{OutBox}

\begin{table}[h!]
    \centering
    \begin{tabular}{ | l | l | l | l | l | }
        \hline
        Описание            & Тип 	                & Владелец  & Группы    & Остальные \\ \hline
        \hline
        Символьная запись   & l 	                & rwx       & rwx       & rwx       \\ \hline
        Что значит          & символическая ссылка  & все права & все права & все права \\ \hline
    \end{tabular}
\end{table}

Просматриваем права \verb|домашнего каталога|.

\begin{BashBox}
    ls -ld ~
\end{BashBox}

\begin{OutBox}
    drwxr-xr-x 29 pavel-innokentevich-galanin pavel-innokentevich-galanin 4096 Mar  1 21:21 /home/pavel-innokentevich-galanin
\end{OutBox}

\begin{table}[h!]
    \centering
    \begin{tabular}{ | l | l | l | l | l | }
        \hline
        Описание            & Тип       & Владелец  & Группы            & Остальные       \\ \hline
        \hline
        Символьная  запись  & d         & rwx       & r-x               & r-x             \\ \hline
        Что значит          & директория& все права & доступ на чтение  & доступ на чтение\\ \hline
    \end{tabular}
\end{table}

Просмотр прав в каталоге /etc

\begin{BashBox}
    ls -l /etc
\end{BashBox}

Больше прав drwxr-xr-x

\begin{table}[h!]
    \centering
    \begin{tabular}{ | l | l | l | l | l | }
        \hline
        Описание            & Тип       & Владелец  & Группы            & Остальные       \\ \hline
        \hline
        Символьная  запись  & d         & rwx       & r-x               & r-x             \\ \hline
        Что значит          & директория& все права & доступ на чтение  & доступ на чтение\\ \hline
    \end{tabular}
\end{table}

Просмотр прав в каталоге /bin

\begin{BashBox}
    ls -l /bin
\end{BashBox}

\begin{OutBox}
    lrwxrwxrwx 1 root root 7 Feb 25 21:42 /bin -> usr/bin
\end{OutBox}

\begin{BashBox}
    ls -l /usr/bin
\end{BashBox}

Больше прав -rwxr-xr-x

\begin{table}[h!]
    \centering
    \begin{tabular}{ | l | l | l | l | l | }
        \hline
        Описание            & Тип   & Владелец  & Группы            & Остальные       \\ \hline
        \hline
        Символьная  запись  & -     & rwx       & r-x               & r-x             \\ \hline
        Что значит          & файл  & все права & доступ на чтение  & доступ на чтение\\ \hline
    \end{tabular}
\end{table}

Просмотр прав в каталоге /home

\begin{BashBox}
    ls -l /home
\end{BashBox}

Больше прав drwxr-xr-x


\begin{table}[h!]
    \centering
    \begin{tabular}{ | l | l | l | l | l | }
        \hline
        Описание            & Тип       & Владелец  & Группы            & Остальные       \\ \hline
        \hline
        Символьная  запись  & d         & rwx       & r-x               & r-x             \\ \hline
        Что значит          & директория& все права & доступ на чтение  & доступ на чтение\\ \hline
    \end{tabular}
\end{table}
